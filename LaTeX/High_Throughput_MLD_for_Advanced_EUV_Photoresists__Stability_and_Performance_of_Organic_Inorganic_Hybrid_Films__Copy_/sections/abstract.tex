\begin{abstract}
As extreme ultraviolet (EUV) lithography continues to push toward sub-\SI{10}{nm} semiconductor patterning, next-generation photoresists must meet increasingly stringent demands for resolution, sensitivity, line-edge roughness, and etch durability. Hybrid organic--inorganic thin films, particularly alucones and zincones deposited via molecular layer deposition (MLD), provide precise tunability of chemical structure, uniformity of deposition, and intrinsic plasma etch resistance, making them promising EUV resist candidates. In this study, we systematically evaluate alucone and zincone films synthesized using a series of bifunctional organic diols to elucidate how structure and composition affect lithographic performance. Film deposition parameters were optimized, and films were comprehensively characterized by spectroscopic ellipsometry, Fourier-transform infrared spectroscopy (FTIR), and X-ray photoelectron spectroscopy (XPS). Stability assessments under ambient conditions, solvent immersion, and controlled ultraviolet (UV) irradiation provided detailed insights into dissolution mechanisms and photochemical cross-linking pathways. Electron-beam lithography experiments validated their resist capabilities, demonstrating sub-\SI{50}{nm} resolution and improved lithographic contrast attributed to UV-induced organic cross-linking. These results establish alucone and zincone systems as versatile platforms for EUV photoresist development, presenting a comprehensive structure--property framework to guide the rational design of advanced lithographic materials.
\end{abstract}
