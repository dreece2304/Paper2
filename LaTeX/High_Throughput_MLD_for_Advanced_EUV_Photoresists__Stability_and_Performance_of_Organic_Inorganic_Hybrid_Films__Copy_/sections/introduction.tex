\section{Introduction}

Extreme ultraviolet (EUV) lithography at \SI{13.5}{\nano\meter} has emerged as the principal exposure technology for advanced-node logic and memory fabrication, routinely enabling half-pitch dimensions below \SI{20}{\nano\meter}~\cite{zhang2016euv,Xu2018}. To fully leverage this capability, photoresist materials must simultaneously achieve high sensitivity \(\leq\) \SI{20}{\milli\joule\per\centi\meter\squared}, low line-edge roughness (LER \(\leq\) \SI{3}{\nano\meter}), precise critical dimension control, and robust plasma etch resistance—criteria that have historically challenged conventional chemically amplified resists (CARs)~\cite{Manouras2020,Mojarad2015,Vesters2018}. 

CAR formulations rely on acid diffusion mechanisms, leading to inherent trade-offs among sensitivity, resolution, and LER, and often fail to deliver sufficient inorganic content for advanced etch processing~\cite{Harumoto2021}. Consequently, significant attention has shifted toward alternative resist platforms, particularly metal-rich and hybrid organic–inorganic materials, which promise improved etch resistance, resolution, and radiation sensitivity~\cite{Cardineau2014Tin,Saifullah2022Review,Ravi2023,Shi2022}.

Early inorganic candidates, including hydrogen silsesquioxane (HSQ) and perhydrogen polysilazane, offered excellent etch durability but required impractically high exposure doses and were prone to swelling during development~\cite{Saifullah2022Review}. More recent approaches utilizing tin-oxo clusters and hafnium oxide nanoparticles have achieved pattern resolutions below \SI{15}{\nano\meter}. However, their reliance on colloidal processes often complicates film uniformity and defectivity control~\cite{Cardineau2014Tin,Saifullah2022Review}.

Hybrid organic–inorganic thin films fabricated via molecular layer deposition (MLD) offer an attractive alternative by combining inorganic durability with tunable organic components, thereby enabling precise control over solubility, cross-link density, and radiation chemistry~\cite{marichy2012mld,meng2017mldreview,Multia2022,Vemuri2023}. Due to the inherently self-limiting, surface-controlled nature of MLD reactions, these films achieve exceptional thickness uniformity, compositional precision, and molecular-level tunability—features difficult to replicate with solution-based processing~\cite{marichy2012mld,Lee2011,Weckman2016}.

Alucones and zincones, hybrid networks generally described by the repeating structural units \(\left[\mathrm{Al-O-R-O}\right]_n\) and \(\left[\mathrm{Zn-O-R-O}\right]_n\), respectively, where \textit{R} is a divalent organic linker derived from bifunctional diols, have emerged as model systems for studying hybrid resist behavior~\cite{Choudhury2015,Perrotta2019}. Molecular layer deposition involves sequential exposures to a metal alkyl precursor (trimethylaluminum for alucones, diethylzinc for zincones) and a bifunctional organic diol (\(\mathrm{HO-R-OH}\)), resulting in a precisely constructed metal–oxygen–carbon hybrid framework. The inorganic M–O backbone (M = Al or Zn) provides mechanical stability and etch resistance, while the organic linker defines free volume, radiation sensitivity, and developer solubility~\cite{Vemuri2023,Lee2014}.

By employing linkers spanning saturated, unsaturated, and aromatic structures, the stereochemical and electronic influences on resist performance can be systematically investigated. However, evaluating such diverse metal–organic combinations is experimentally intensive, as each system requires independent optimization of precursor dosing, purge timing, and thermal conditions. Traditional single-chamber reactors impose prohibitive time constraints for such combinatorial studies. 

To address this limitation, we developed a high-throughput MLD reactor featuring six independent deposition chambers, enabling simultaneous growth of distinct chemistries with undetectable cross-contamination~\cite{Choe2024}. Using this platform, alucone and zincone films incorporating ethane-1,2-diol (EG), 2-methylene-1,3-propanediol (MPD), 3,4-dihydroxy-1-butene (DHB), cis-2-butene-1,4-diol (CB), 1,4-butynediol (BTY), and 1,3,4-trihydroxybenzene (THB) were systematically synthesized to elucidate the effects of organic linker chemistry on resist properties.

Three critical performance metrics were assessed: (i) air stability via in situ ellipsometry, (ii) developer compatibility across organic solvents and aqueous solutions, and (iii) lithographic performance using electron-beam lithography (EBL). Complementary Fourier-transform infrared spectroscopy (FTIR) and X-ray photoelectron spectroscopy (XPS) analyses were performed to probe chemical transformations such as bond cleavage, cross-linking, and structural reorganization within the hybrid networks.

Collectively, this comprehensive evaluation establishes detailed structure–property correlations linking organic linker chemistry, metal center selection, and lithographic performance. The mechanistic insights derived herein lay the groundwork for rational design of next-generation hybrid EUV photoresists tailored to meet the stringent demands of sub-\SI{10}{\nano\meter} semiconductor patterning.
